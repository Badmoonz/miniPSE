\documentclass[10pt,pdf,hyperref={unicode}]{beamer}
\usepackage[T2A]{fontenc}       %поддержка кириллицы
\usepackage[utf8]{inputenc}   %пока бибтех не дружит до конца с юникодом
\usepackage[russian]{babel}     %определение языков в документе
\usepackage{amssymb,amsmath}    %математика

\graphicspath{{./pictures/}{../../pictures/}} %относительный путь к
                                %каталогу с рисунками (обязателен слеш
                                %в конце!)

% Тема презентации
\usetheme[numbers, totalnumbers, minimal, nologo]{Boadilla}

%%%%%%%%%%%%%%%%%%%
%% Выбор шрифтов %%
\usefonttheme[onlylarge]{structurebold}

% Привычный шрифт для математических формул
\usefonttheme[onlymath]{serif}

% Более крупный шрифт для подзаголовков титульного листа
\setbeamerfont{institute}{size=\normalsize}
%%%%%%%%%%%%%%%%%%%

% Если используется последовательное появление пунктов списков на
% слайде (не злоупотребляйте в слайдах для защиты дипломной работы),
% чтобы еще непоявившиеся пункты были все-таки немножко видны.
\setbeamercovered{transparent}

%%%%%%%%%%%%%%%%%%
%%% Сокращения %%%
% Синий цвет выделения
\setbeamercolor{color1}{bg=blue!60!black,fg=white}
\newcommand{\celcius}{\,^{\circ}\mathrm{C}}  %градус Цельсия
\newcommand{\grad}{\,^{\circ}}               %знак градуса
%%%%%%%%%%%%%%%%%%

\title{Формализация и анализ гибридных систем потока работ  для решения 
научных задач}
\author{Козлов Алексей}
\institute{НИУ ВШЭ
%    \vspace{0.7cm}
 %   Научный руководитель:  ФИО шефа с регалиями \\
%    \vspace{0.7cm}
}
\date{
    \\
    2013г.
}

\begin{document}
\begin{frame}
  % создаём титульный лист
  \maketitle
\end{frame}

\section{Введение}

\begin{frame}
  \frametitle{Понятие workflow}
  \begin{columns}
    % Колонки по половине ширины слайда
    \column{0.5\textwidth}
    \only<1->{
    \textbf{Термин “workflow” используется в двух аспектах}:
    \begin{itemize}
        \item<1-2,5> формальное представление некоторого процесса
        \item<1,3-> подход к автоматизации процессов, основанный на подобном представлении
    \end{itemize}    
    }

    \column{0.5\textwidth}
    \only<2>{
\begin{enumerate}
\item[-] Описание элементарных операций, из которых состоит процесс.
\item[-] Описание исполнителей, которые выполняют указанные операции.
\item[-] Описание зависимостей между операциями, а именно — потоков управления и потоков данных.
\item[-] Описание внешних событий, которые могут влиять на ход процесса,
и правил их обработки.
\end{enumerate}
 }
 \only<3>{
 «это автоматизация бизнес-процесса, при которой документы, информация или задания передаются для выполнения необходимых действий
от одного участника к другому в соответствии с набором процедурных
правил» 
}
 \only<4>{
Системой управления  (workflow management system, WFMS)
называется система, позволяющая:
\begin{enumerate}
\item[•] создавать сценарии
процессов
\item[•] запускать
и управлять  выполнением

\end{enumerate} 
WFMS состоит из набора программных
компонентов, предназначенных для:
\begin{enumerate}
\item[•] хранения и интерпретации описаний
процессов 
\item[•] создания и управления экземплярами запущенных процессов
\item[•] организации их взаимодействия с участниками
процесса и внешними приложениями.
\end{enumerate} 
}
 \only<5>{
Программное приложение, непосредственно выполняющее интерпретацию и запуск сценария, а также управляющее экземплярами запущенных процессов, будем называть средой выполнения сценариев (workflow engine).
 }
 

  \end{columns}
\end{frame}



\end{document}