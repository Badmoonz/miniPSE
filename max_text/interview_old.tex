\documentclass[a4paper,14pt]{article}
\usepackage{amsmath,amsfonts,amssymb,amsthm,epsfig,epstopdf,titling,url,array}
\usepackage[utf8x]{inputenc}
\usepackage[russian]{babel}
%\usepackage[T2A]{fontenc}
\usepackage{amsmath,amssymb,amsthm,amscd,amsfonts,graphicx}
\usepackage[14pt]{extsizes}

\begin{document}

\textwidth 15.5cm
\topmargin -1cm
\parindent 1cm
\textheight 24cm
\parskip 1.5mm
\begin{titlepage}
Введение.
\end{titlepage}



\section{The Need for Composing Models of Computation in E-science}

\section{Workflows and Hierarchy}
Описание понятия  	акторa(actor,atomic actor.
,composite actor (a.k.a. sub-workflow))
\section{(Модели управления потоком)Models of Computation}
\subsection*{Process Networks (PN)}
\subsection{Использование PN}
\begin{enumerate}
\item[•] Тупики
\item[•] 
\end{enumerate}
Более подробно модель описывается  через Kahn networks.
Каждый актор запускается в отдельно треде(thread)(или вычислительном узле), и все акторы запускаются в конкурирующем режиме. 
\section*{SDF}
\subsection{Задачи}
\subsection{Использование SDF}
\begin{enumerate}
\item[] Deadlocks
\item[•] Consistency of data rates
\item[•] The value of the iterations parameter
\item[•] The granularity of execution
 Более подробно в секции 3.3.
\end{enumerate}
\subsection{Тупики(Deadlocks)}
\subsection{Consistency of data rates}
Все акторы внутри \textit{однородного(homogeneous)} SDF принимают только по одному токену из каждого входного порта и выписывают по 1 токену в наждый выходно порт. В кадом цикле должить быть "сдерживающий" актор,  так что dataflow-граф будет ациклическим , если убрать все "сдерживающие" акторы. Схему запуска можно определить статически, например, через топологическая сортировка графа.
%\subsubsection{Количество итераций}
\subsection{Свойства SDF}
\par SDF не связан с временными событяиями. Для всех акторов,входящих в SDF,  поглощение токенов из входных портов, выполнение вычисленых операций и отправление токенов в выходные порты явсляется атомарной операцией. Запуск композитного актора соответствует одной итерации содержащейся в нём модели по предварительно вычисленной схеме(schedule). При это схема выполнения рассчитывается так, чтобы , и при бесконечном числе итераций в модели не возникали тупики и накопление токенов. 
\subsubsection{Вычисление схем запуска}
Вложенные друг в друга SDF модели , можно уплощать(flasttern),
Запуск последовательный, преимущзественно на одном узле.

\section*{FSM}
Конечные автоматы (Finite State Machines) 
\section*{DDF}


В этой модели управления вместо того, чтобы выдавать каждому актору отдельный тред, система управления запускает отдельный актор, когда необходимые удовлетворены его зависимости на входе.  Dataflow делятся на два типа, dynamic dataflow (DDF) и synchronous dataflow (SDF). В случае DDF, система управления динамически определяет  какой актор необходимо запустить в следующим, и следовательно составляет схему запуска(firing schedule) динамически во время работу. В случае SDF, сиситема управления 
\begin{enumerate}
\item[•] Сети Петри
\item[•] Анализ Сетей Петри
\item[•] Граф Карпа и Миллера 
\item[•] Marked Graphs
\item[•] Однородные dataflow
\item[•] Обобщённые dataflow
\item[•] Модели Kaна (Khan) для параллельных вычислений
\end{enumerate}

9.0  Специальная лексика
 Dataflow - поток данных 


10.0 Используема литература

\section{СЕТИ}
\subsection{Сети Петри}
Сети Петри широко используются для моделирования и исcледования динамических дискретных систем. 
И прежде чем рассмотреть рассмотреть частные случаи использования Сетей Петри, приведём описание их каноничной формы, согласно определению Петерсона [Pet81].
\par Сеть Петри представляет собой двудольный ориентированный граф, состоящий из вершин двух типов — позиций и переходов, соединённых между собой дугами. Вершины одного типа не могут быть соединены непосредственно. В позициях могут размещаться метки (маркеры), способные перемещаться по сети.\\
Простой сетью Петри наызвается набор $N = (S,T,F)$ , где
\begin{enumerate}
\item $S = \lbrace s_{1},\ldots,s_{n} \rbrace$ - множество \textit{позиций}
\item $T = \lbrace t_{1},\ldots,t_{r} \rbrace$ - множество \textit{переъходов} таких, что $S \bigcap T = \varnothing$.
\item $F \subseteq \mu S \times T \times \mu S$ - отношение \textit{инцидентности} такое, что 
\begin{enumerate}
\item[•] $\forall \langle  Q_{1}^{'}, t_{1}, Q_{1}^{''} \rangle , \langle Q_{2}^{'}, t_{1}, Q_{2}^{''}\rangle \in F : \langle Q_{1}^{'}, t_{1}, Q_{1}^{''} \rangle \neq \langle Q_{2}^{''}, t_{2}, Q_{2}^{'}\rangle \Rightarrow t_{1} \neq t_{2};$
\item[•] $\lbrace t | \langle  Q_{1}^{'}, t_{1}, Q_{1}^{''} \rangle \in F \rbrace = T$
\end{enumerate}
\end{enumerate} 
Условия в пункте 3 говорят , что для каждого перехода $t \in T$ существует идинственный элемент $\langle Q^{'}, t, Q^{''} \rangle$, задающий для него входное мультимножество $Q^{'}$ и  выходное мультимножество $Q^{''}$. Дадим определение входному и выходному мультимножеству.

\textbf{Определение:}  \textit{Входное и выходное мультимножества мест и переходов}
\par Пусть задана сеть $N = (S,T,F)$.
\begin{enumerate}
\item Если для некоторого перехода t имеем $\langle Q^{'}, t, Q^{''} \rangle \in F$ , то будем обозначать $I(t) = Q^{'} = \langle (s,n)|(t,n) \in O(s) \rangle , O(t) = Q^{''} = \langle(s,n) | (t ,n) \in I(s) \rangle$
\item  И соответтственно $I(s) = \langle (t,n)|(s,n) \in O(t) \rangle , O(s) = \langle(t,n) | (s ,n) \in I(t) \rangle$
\end{enumerate}
Будем говорить, что I(t) - входные , а O(t) - выходные позиции перехода t. ТАким образом, соласно определению, справедливо $\forall t \in T : \langle I(t), t, O(t) \in F \rangle$. Далее будем говорить, что позиция s инцидентна переходу t , если $s \in I(t) или s \in O(t)$.

Сети Петри имеют удобную графическую форму представления в виде графа, в котором места изображаются кружками, а переходы прямоугольниками. Места и переходы, причем место s соединяется с переходом t если $(s,n) \in I(t)$ и t соеднияется с s если $(s,n) \in O(t)$  для некоторого натурального числа $n \in N$ . Здесь число n называется кратностью дуги, которое графически изображается рядом с дугой. Дуги, имеющие единичную кратность, будут обозначаться без приписывания единицы.

Само по себе понятие сети имеет статическую природу. Для задания динамических характеристик используется понятие маркировки сети $M \i \mu S$, т.е. функции $M : S \longrightarrow N_{0}$, сопоставляющей каждому месту целое число. Графически маркировка изображается в виде точек, называемых метками (tokens), и располагающихся в кружках, соответствующих местам сети. Отсутствие меток в некотором месте говорит о нулевой маркировке этого места.

\textbf{Определение:}  \textit{Маркированная сеть Петри}
Маркированной сетью Петри называется набо $\Sigma = (S,T,F M_{0})$, где
\begin{enumerate}
\item (S,T,F) - сеть;
\item $M_{0} \in  \mu S$ - наяальная маркировка.
\end{enumerate}

Работа мариковочной сети Петри управляется наличием или отсутствием маркировочных токенов. Сеть Петри срабатывает  переход t, в процессе которого c каждой  мультипликативного входа $(s, n) \in I(t)$ снимается по n токенов с каждого, где n - соответствующая кратность дуги и каждому мультипликативного выходу $(s, m) \in O(t)$ прибавляется m токенов.
Для любого положения количество токенов не может быть отрицательным, поэтому переъод не может сработать, если количство токенов на входах меньше требуемого количества токенов на выходе. 

\textbf{Опеределение: } \textit{Правило срабатывания переходов}
Пусть $\Sigma = (S,T,F M_{0})$ маркировочная сеть.
\begin{enumerate}
\item Переход $t \in T$ считается возбуждённым при маркировке $M \in \mu S$ если $M \geq I(t)$;
\item переход t, возбуждённый при маркировке M,  может сработать, приведя к новой маркировке$M^{'}$,  которая вычисляется по правилу: $M^{'} = M - I(t) + O(t)$. Срабатывание перехода обозначается 
\end{enumerate}


\textbf{ПРИМЕР}

\subsection{Анализ сетей Петри}
Сети петри могут быть использованя для моделирования конкурирующих систем. К примеру, сеть процессов с общей памятью. 

\par Композициональный подход к построению сетей Петри предполагает возможность построения более сложных сетей из менее сложных составляющих. Для этого вводятся точки доступа, которые позволяют объединять простые сети путём синхронизации событий и состояний (переходов и мест).
\par Обычно в сетях Петри считается, что если при одной и той же маркировке возбуждено несколько переходов, то может сработать любой, но только один из них. Это ограничение не является принципиальным и может быть снято.
При применении сетей Петри для целей управления позициям сопос­
тавляются операции (действия), а переходам — условия, при выполнении
которых возбужденные переходы срабатывают, активизируя соответству­ ющие операции. При этом попадание меток в позицию ассоциируется с началом операции, а удаление метки — с ее окончанием. При использовании такого предположения считают, что любая операция не может быть повторно начата до ее завершения. Для описания таких процессов могут применяться только безопасные сети петри, т. е. такие сети, в которых при любой начальной маркировке $\mu$ невозможно ни через какую последовательность выполненных переходов получить такую маркировку $\mu^{'}$ с количеством токенов в положении больше единицы.
\par Возможно сделать сеть Петри быть безопасной добавляя дуги, обеспечивая 
Безопасность 







\begin{thebibliography}{10}
\bibitem{GG}[GG] Gray L., Griffeath D. The ergodic theory of traffic jams // J. Stat. 
\end{thebibliography}{10}

\end{document}
